\documentclass[12pt,english]{article}
\usepackage[T1]{fontenc}
\usepackage[latin9]{inputenc}
\usepackage{geometry}
\geometry{verbose,tmargin=2.5cm,bmargin=2.5cm,lmargin=2.5cm,rmargin=2.5cm}
\usepackage{setspace}
\usepackage[round,authoryear]{natbib}
\onehalfspacing

\makeatletter

%%%%%%%%%%%%%%%%%%%%%%%%%%%%%% LyX specific LaTeX commands.
%% Because html converters don't know tabularnewline
\providecommand{\tabularnewline}{\\}

%%%%%%%%%%%%%%%%%%%%%%%%%%%%%% User specified LaTeX commands.
\usepackage[hidelinks]{hyperref}

\makeatother

\usepackage{babel}
\begin{document}
	\begin{center}
		{\Large{}README}{\Large\par}
		\par\end{center}
	
	\begin{center}
		{\Large{}Multidimensional Auctions of Contracts:}\\
		{\Large{}An Empirical Analysis}\\
		{\Large{}Repository: openicpsr-154601}{\Large\par}
		\par\end{center}
	
	\begin{center}
		Yunmi Kong \hspace*{1cm} Isabelle Perrigne \hspace*{1cm} Quang Vuong
		\vspace{1cm}
		\par\end{center}
	
	\subsection*{Data }
	
	\subsubsection*{Data availability}
	
	All data used in the paper are provided in the replication package
	with the following exceptions: historical crude oil futures prices
	from Quandl, historical crude oil options prices from the CME Group,
	and oil production data from Drillinginfo require an account and/or
	purchase from the respective data providers and are not provided here.
	Replicators may obtain these data from the respective data providers.
	Meanwhile, the replication package provides sufficient derived analysis
	to enable replication of the paper using only what is provided in
	the package.
	
	\subsubsection*{Data sources}
	
	$\circ$ Main data
	\begin{itemize}
		\item Louisiana Department of Natural Resources (DNR)
		\begin{itemize}
			\item Publicly available state government records
			\begin{itemize}
				\item \citet{dnrauc} Tract sheets of auction results. Accessed April 2017 at \\
				http://reports.dnr.state.la.us/reports/rwservlet?SRMN9031B\_p
				\item \citet{dnrlse} Lease information. Accessed September 2017 at\\
				http://sonlite.dnr.state.la.us/sundown/cart\_prod/cart\_min\_qld1
			\end{itemize}
		\end{itemize}
	\end{itemize}
	$\circ$ Auxiliary data
	\begin{itemize}
		\item \citet{fredwti} %Federal Reserve Bank of St. Louis
		\begin{itemize}
			\item Publicly available federal reserve bank statistics
			\begin{itemize}
				\item Spot Crude Oil Price: West Texas Intermediate (WTISPLC). Accessed August 2019 at\\
				https://fred.stlouisfed.org/series/WTISPLC 
			\end{itemize}
		\end{itemize}		
		\item \citet{bea} %Bureau of Economic Analysis (BEA)
		\begin{itemize}
			\item Publicly available federal bureau statistics
			\begin{itemize}
				\item GDP implicit price deflator, in National Income and Product Accounts
				table 1.1.9. Accessed October 2017 at\\
				https://apps.bea.gov/iTable/iTable.cfm?reqid=19\&step=3\&isuri=1\&1921=sur vey\&1903=13\#reqid=19\&step=3\&isuri=1\&1921=survey\&1903=13
			\end{itemize}
		\end{itemize}
		\item \citet{fredgs1} (FRB) %Board of Governors of the Federal Reserve System (US)
		\begin{itemize}
			\item Publicly available federal reserve bank statistics
			\begin{itemize}
				\item 1-Year Treasury Constant Maturity Rate, Percent, Monthly, Not Seasonally
				Adjusted. Accessed March 2018 at\\
				https://fred.stlouisfed.org/series/GS1
			\end{itemize}
		\end{itemize}	
		\item \citet{quandl} (was Quandl at time of access, now Nasdaq Data Link)
		\begin{itemize}
			\item Data access requires an account (free). Accessed April 2019.\\
			https://data.nasdaq.com
			\begin{itemize}
				\item Historical Crude Oil Futures Prices (Data product: Wiki Continuous
				Futures, Frequency: daily, Nasdaq data link code: CHRIS/CME\_CL1 through
				CHRIS/CME\_CL36), up to April 22, 2019. 
			\end{itemize}
		\end{itemize}
		\item \citet{cme} %Chicago Mercantile Exchange (CME)
		\begin{itemize}
			\item Data must be purchased from the CME Group. Accessed January 2019.\\
			https://www.cmegroup.com
			\begin{itemize}
				\item Crude oil options complete historical (product category: EOD, product
				name: Crude Oil Options, symbol: LO, order type: Complete Historical).
			\end{itemize}
		\end{itemize}
		\item \citet{drillinginfo} (was Drillinginfo at time of access, now Enverus)
		\begin{itemize}
			\item Data access requires an account. Accessed March 2018.\\
			https://www.enverus.com
			\begin{itemize}
				\item Monthly production data for Louisiana wells, up to January 2018. Steps
				for access: once logged in, apply Data Filters > Filter > State >
				LA. Among the datasets, select ``Production''. Click ``Apply''.
				Then go to the Exports tab, select ``Production'', then click ``Next''.
				Select ``Production Headers.csv'' and ``Producing Entity Monthly
				Production.csv'', click ``Next'', and complete the data export.
			\end{itemize}
		\end{itemize}
	\end{itemize}
	
	\subsubsection*{Dataset list}
	
	\begin{tabular}{llll}
		& \textbf{Data file} & \textbf{Provided} & \textbf{Source}\tabularnewline
		{[}1{]} & data/maindata.dta & Yes & DNR, St. Louis Fed\tabularnewline
		{[}2{]} & data/townships.dta & Yes & DNR\tabularnewline
		{[}3{]} & data/gdpipdeflatorQ.dta & Yes & BEA\tabularnewline
		{[}4{]} & data/treasury\_1y\_monthly.dta & Yes & FRB\tabularnewline
		{[}5{]} & data/quandl/CHRIS-CME\_CL`m'.csv & No & Quandl\tabularnewline
		{[}6{]} & data/cme/cme\_lo`date'.csv & No & CME\tabularnewline
		{[}7{]} & data/Producing Entity Monthly Production.csv & No & Drillinginfo\tabularnewline
		{[}8{]} & data/Production Headers.csv & No & Drillinginfo\tabularnewline
		{[}9{]} & calculations/tractnum\_iv.dta & Yes & Derived from {[}1{]},{[}5{]},{[}6{]}\tabularnewline
		{[}10{]} & calculations/twp\_idx\_pre.dta & Yes & Derived from {[}2{]},{[}7{]},{[}8{]}\tabularnewline
		{[}11{]} & calculations/twp\_prod\_post.dta & Yes & Derived from {[}2{]},{[}7{]},{[}8{]}\tabularnewline
		{[}12{]} & calculations/twp\_aveprod\_1987\_2006.dta & Yes & Derived from {[}2{]},{[}7{]},{[}8{]}\tabularnewline
	\end{tabular}
	
	\smallskip
	\begin{itemize}
		\item Notes
		\begin{itemize}
			\item {[}5{]} and {[}6{]} are not needed for replication given {[}9{]}.
			\item {[}7{]} and {[}8{]} are not needed for replication given {[}10{]},
			{[}11{]}, {[}12{]}.
		\end{itemize}
	\end{itemize}
	
	\subsection*{Computational requirements}
	
	\subsubsection*{Software requirements}
	\begin{itemize}
		\item Matlab (code was run with Matlab Release 2020a)
		\begin{itemize}
			\item Curve Fitting Toolbox
			\item Optimization Toolbox
			\item Statistics and Machine Learning Toolbox
		\end{itemize}
		\item R 4.1.1
		\begin{itemize}
			\item quantmod (0.4.18)
			\item There is only one R file; it includes installation of dependencies.
			\item Note: R is used only in preparing data file {[}5{]} (see dataset list)
			which is not provided. R is not needed when running the replication
			with what is provided in this package. 
		\end{itemize}
		\item STATA (code was run with STATA SE 15.1)
		\begin{itemize}
			\item estout (as of 2020-02-03)
			\item outreg2 (as of 2020-02-03)
		\end{itemize}
	\end{itemize}
	
	\subsubsection*{Memory and runtime requirements}
	
	The code was run on a 56-core Intel-based desktop with 512 GB of RAM,
	2 TB of fast local storage. Some Matlab programs contain \texttt{parfor}
	(parallel for-loop) commands, so the number of cores substantially
	affects run times. Run times were as follows.
	
	\vspace{0.5cm}
	
	\noindent%
	\begin{tabular}{lll}
		\hline 
		Program & Software & Time\tabularnewline
		\hline 
		A\_run\_stata\_code.do & Stata & 1 sec\tabularnewline
		matlab/B\_heatmap3.m & Matlab & 33 sec\tabularnewline
		C\_run\_stata\_code.do & Stata & 70 sec\tabularnewline
		D\_run\_matlab\_code.m & Matlab & 3 hrs\tabularnewline
		E\_run\_appendix\_matlab\_code.m & Matlab & 1 hr 8 min \tabularnewline
		\hline 
	\end{tabular}
	
	\subsection*{Description of programs/code}
	\begin{itemize}
		\item The file \texttt{0\_setup.do} installs necessary Stata packages and
		sets a global macro for the Stata working directory.
		\item Programs in \texttt{stata/prep} will prepare auxiliary variables,
		merge them with the main data, and export data for use in Matlab programs.
		The files \texttt{A\_run\_stata\_code.do} and \texttt{C\_run\_stata\_code.do}
		will run them in the correct order.
		\item Programs in \texttt{stata/tables} will generate Tables 1 and 3 in the main manuscript and Table A1 in the appendix. The file \texttt{C\_run\_stata\_code.do} will run these.
		\item Among programs in \texttt{matlab}, file names starting in `E' will
		estimate the model of the paper, file names starting in `C' will perform
		counterfactual simulations, and file names starting with `figure'
		or `table' will generate figures and/or tables. The file \texttt{D\_run\_matlab\_code.m}
		will run them in the correct order to replicate the main manuscript. The file \texttt{B\_heatmap3.m}
		is run separately as part of data preparation, per the Instructions
		to replicators below.
		\item The file \texttt{E\_run\_appendix\_matlab\_code.m} will run Matlab files in the correct order to replicate the appendix.
	\end{itemize}
	
	\subsection*{Instructions to replicators}
	
	After downloading the replication folder in its entirety without changing,
	moving, or renaming its contents, follow these steps in strict sequence:
	\begin{enumerate}
		\item Edit \texttt{0\_setup.do} to adjust the path to the downloaded replication
		folder, and run it.
		\item Run \texttt{A\_run\_stata\_code.do} in Stata.
		\item Run \texttt{matlab/B\_heatmap3.m} in Matlab.
		\item Run \texttt{C\_run\_stata\_code.do} in Stata.
		\item Run \texttt{D\_run\_matlab\_code.m} in Matlab.
		\item To replicate the appendix, run \texttt{E\_run\_appendix\_matlab\_code.m} in Matlab.
	\end{enumerate}
	\begin{itemize}
		\item Details
		\begin{itemize}
			\item Order matters. The steps should be followed in sequence. If running
			programs individually, they should be run in the order listed in the
			master files above.
			\item The steps above allow replication of the paper's analysis using only
			what is provided in the package. To replicate starting with the data
			files {[}5{]}-{[}8{]} (see dataset list) which are not provided, uncomment
			and run the programs in lines 13-36 of \texttt{A\_run\_stata\_code.do}.
			\item \texttt{calculations} is a folder in which intermediate calculations
			will be deposited.
			\item \texttt{output} is a folder in which figures and tables will be deposited.
			\item The steps above were last run top to bottom in November 2021.
		\end{itemize}
	\end{itemize}
	
	\subsection*{List of tables and programs}
	\subsubsection*{Main manuscript}
	\begin{tabular}{lll}
		\hline 
		Figure/Table \# & Program & Output file (in `output' folder)\tabularnewline
		\hline 
		Table 1 & stata/tables/table1.do & Table1a.xml, Table1b.xml\tabularnewline
		Table 2 & n/a; Table 2 is not empirical. & n/a\tabularnewline
		Table 3 & stata/tables/table3.do & Table3.tex\tabularnewline
		Table 4 & matlab/figure8\_table\_4\_5.m & Table4.csv\tabularnewline
		Table 5 & matlab/figure8\_table\_4\_5.m & Table5.csv\tabularnewline
		Figure 1 & matlab/figure1.m & figure1.eps\tabularnewline
		Figure 2 & matlab/figure2.m & figure2.eps\tabularnewline
		Figure 3 & matlab/E2\_Estimate\_gchoice.m & figure3.eps\tabularnewline
		Figure 4 & matlab/figure4.m & figure4.eps\tabularnewline
		Figure 5 & matlab/figure5\_6\_7.m & figure5.eps\tabularnewline
		Figure 6 & matlab/figure5\_6\_7.m & figure6.eps\tabularnewline
		Figure 7 & matlab/figure5\_6\_7.m & figure7.eps\tabularnewline
		Figure 8 & matlab/figure8\_table\_4\_5.m & figure8.eps\tabularnewline
		\hline 
	\end{tabular}

\subsubsection*{Appendix}

	\begin{tabular}{lll}
		\hline 
		Figure/Table \# & Program & Output file (in `output' folder)\tabularnewline
		\hline 
		Table A1 & stata/tables/tableA1.do & TableA1.tex\tabularnewline
		Table A2 & matlab/tableA2.m & TableA2.csv\tabularnewline
		Table A3 & matlab/tableA3.m & TableA3.csv\tabularnewline
		Table A4 & matlab/figureA8\_table\_A4\_A5.m & TableA4.csv\tabularnewline
		Table A5 & matlab/figureA8\_table\_A4\_A5.m & TableA5.csv\tabularnewline
		Figure A1 & matlab/figureA1toA4.m & figureA1.eps\tabularnewline
		Figure A2 & matlab/figureA1toA4.m & figureA2.eps\tabularnewline
		Figure A3 & matlab/figureA1toA4.m & figureA3.eps\tabularnewline
		Figure A4 & matlab/figureA1toA4.m & figureA4.eps\tabularnewline
		Figure A5 & matlab/C16\_Increase\_t\_compare\_am.m & figureA5.eps\tabularnewline
		Figure A6 & matlab/C16\_Increase\_t\_compare\_am.m & figureA6.eps\tabularnewline
		Figure A7 & matlab/C09\_Increase\_p\_compare.m & figureA7.eps\tabularnewline
		Figure A8 & matlab/figureA8\_table\_A4\_A5.m & figureA8.eps\tabularnewline
		Figure A9 & matlab/figureA9toA15.m & figureA9.eps\tabularnewline
		Figure A10 & matlab/figureA9toA15.m & figureA10.eps\tabularnewline
		Figure A11 & matlab/figureA9toA15.m & figureA11.eps\tabularnewline
		Figure A12 & matlab/figureA9toA15.m & figureA12.eps\tabularnewline
		Figure A13 & matlab/figureA9toA15.m & figureA13.eps\tabularnewline
		Figure A14 & matlab/figureA9toA15.m & figureA14.eps\tabularnewline
		Figure A15 & matlab/figureA9toA15.m & figureA15.eps\tabularnewline
		Figure A16 & matlab/figureA16.m & figureA16.eps\tabularnewline		
		\hline 
	\end{tabular}

%\bibliographystyle{aea}
%\bibliography{Bibdatabase_KPV}

\end{document}
